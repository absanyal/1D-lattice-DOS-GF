\documentclass[twoclumn, a4, 11pt]{article}
\usepackage{amsmath}
\usepackage{amsfonts}
\usepackage{amssymb}
\usepackage{amsthm}
\usepackage{enumerate}
\usepackage{booktabs}
\usepackage{inputenc}
\usepackage{physics}
\usepackage{graphicx}
\usepackage{geometry}
\usepackage{bm}
\usepackage{fancyhdr}
\usepackage{textcomp}
\usepackage{lipsum}
\usepackage{longtable}
\usepackage{adjustbox}
\usepackage{siunitx}

\geometry{top = 2.5cm, bottom = 2.5cm, left = 2.5cm, right = 2.5cm}

\setlength{\parindent}{0mm}
\setlength{\parskip}{0.3em}
\setcounter{secnumdepth}{3}

\renewcommand\refname{Citations and References}
\renewcommand{\abstractname}{\vspace{-\baselineskip}}

\pagestyle{fancy}
\fancyhf{}
\rhead{Page \thepage}
\lhead{Critical Potentials for Anti-binding State Formation}

\begin{document}
\twocolumn[{
\begin{center}
{\LARGE \textbf{Critical Potentials for Anti-binding State Formation}}\\
\today
\end{center}
}]

The interaction term has been taken as
\begin{equation}
U = U_1 \frac{\exp(- \alpha m ^ 2)}{m ^ \beta}
\end{equation}
where $m$ is the separation between the two particles, and $\alpha$ and $\beta$ are two parameters.

For $\beta=1$ and $\beta = 6$, various values of $\alpha$ have been selected, and, for each pair of $\alpha$ and $\beta$, the $U_1$ required for an anti-binding state formation has been found and plotted against $\alpha$ for a fixed $\beta$.

For $\beta = 1$,
\begin{table}[h!]
\centering
\begin{tabular}{c|c}
\hline\hline
$\alpha$	&	$U_1$	\\
\hline
0	&	4	\\
1	&	12	\\
2	&	30	\\
3	&	80	\\
4	&	220	\\
5	&	600	\\
\hline
\end{tabular}
\caption{Variation of $\alpha$ and $U_1$.}
\end{table}

\begin{figure}[h!]
\centering
\includegraphics[width=1\linewidth]{b_1_plot}
\caption{Variation of $\alpha$ and $U_1$.}
\label{fig:b1plot}
\end{figure}

\pagebreak
For $\beta = 2$,
\begin{table}[h!]
\centering
\begin{tabular}{c|c}
\hline\hline
$\alpha$	&	$U_1$	\\
\hline
0.0	&	4.00	\\
0.2	&	4.75	\\
0.4	&	5.75	\\
0.6	&	7.00	\\
0.8	&	8.75	\\
1.0	&	10.25	\\
1.2	&	12.75	\\
1.4	&	15.25	\\
1.6	&	18.50	\\
1.8	&	22.50	\\
2.0	&	27.50	\\
\hline
\end{tabular}
\caption{Variation of $\alpha$ and $U_1$.}
\end{table}

\begin{figure}[h!]
\centering
\includegraphics[width=1\linewidth]{b_6_plot}
\caption{Variation of $\alpha$ and $U_1$.}
\label{fig:b6plot}
\end{figure}

The variations were fit against an exponential function.

\pagebreak
\begin{figure}[t!]
\centering
\includegraphics[width=1\linewidth]{Critical_Potential_b_1/a_0_b_1}
\caption{Anti-binding state for $\beta = 1$, $\alpha = 0$.}
\label{fig:a0b1}
\end{figure}

\begin{figure}[t!]
	\centering
	\includegraphics[width=1\linewidth]{Critical_Potential_b_1/a_1_b_1}
	\caption{Anti-binding state for $\beta = 1$, $\alpha = 1$.}
	\label{fig:a1b1}
\end{figure}

\begin{figure}[t!]
	\centering
	\includegraphics[width=1\linewidth]{Critical_Potential_b_1/a_2_b_1}
	\caption{Anti-binding state for $\beta = 1$, $\alpha = 2$.}
	\label{fig:a2b1}
\end{figure}

\begin{figure}[t!]
	\centering
	\includegraphics[width=1\linewidth]{Critical_Potential_b_1/a_3_b_1}
	\caption{Anti-binding state for $\beta = 1$, $\alpha = 3$.}
	\label{fig:a3b1}
\end{figure}

\begin{figure}[t!]
	\centering
	\includegraphics[width=1\linewidth]{Critical_Potential_b_1/a_4_b_1}
	\caption{Anti-binding state for $\beta = 1$, $\alpha = 4$.}
	\label{fig:a4b1}
\end{figure}

\begin{figure}[t!]
	\centering
	\includegraphics[width=1\linewidth]{Critical_Potential_b_1/a_5_b_1}
	\caption{Anti-binding state for $\beta = 1$, $\alpha = 5$.}
	\label{fig:a5b1}
\end{figure}

\pagebreak
\begin{figure}[t!]
\centering
\includegraphics[width=1\linewidth]{Critical_Potential_b_6/a_0_b_6}
\caption{Anti-binding state for $\beta = 6$, $\alpha = 0$.}
\label{fig:a0b6}
\end{figure}

\begin{figure}[t!]
\centering
\includegraphics[width=1\linewidth]{Critical_Potential_b_6/a_0p2_b_6}
\caption{Anti-binding state for $\beta = 6$, $\alpha = 0.2$.}
\label{fig:a0p2b6}
\end{figure}

\begin{figure}[t!]
	\centering
	\includegraphics[width=1\linewidth]{Critical_Potential_b_6/a_0p4_b_6}
	\caption{Anti-binding state for $\beta = 6$, $\alpha = 0.4$.}
	\label{fig:a0p4b6}
\end{figure}

\begin{figure}[t!]
	\centering
	\includegraphics[width=1\linewidth]{Critical_Potential_b_6/a_0p6_b_6}
	\caption{Anti-binding state for $\beta = 6$, $\alpha = 0.6$.}
	\label{fig:a0p6b6}
\end{figure}

\begin{figure}[t!]
	\centering
	\includegraphics[width=1\linewidth]{Critical_Potential_b_6/a_0p8_b_6}
	\caption{Anti-binding state for $\beta = 6$, $\alpha = 0.8$.}
	\label{fig:a0p8b6}
\end{figure}

\begin{figure}[t!]
	\centering
	\includegraphics[width=1\linewidth]{Critical_Potential_b_6/a_1p0_b_6}
	\caption{Anti-binding state for $\beta = 6$, $\alpha = 1.0$.}
	\label{fig:a1p0b6}
\end{figure}

\begin{figure}[t!]
	\centering
	\includegraphics[width=1\linewidth]{Critical_Potential_b_6/a_1p2_b_6}
	\caption{Anti-binding state for $\beta = 6$, $\alpha = 1.2$.}
	\label{fig:a1p2b6}
\end{figure}

\begin{figure}[t!]
	\centering
	\includegraphics[width=1\linewidth]{Critical_Potential_b_6/a_1p4_b_6}
	\caption{Anti-binding state for $\beta = 6$, $\alpha = 1.4$.}
	\label{fig:a1p4b6}
\end{figure}

\begin{figure}[t!]
	\centering
	\includegraphics[width=1\linewidth]{Critical_Potential_b_6/a_1p6_b_6}
	\caption{Anti-binding state for $\beta = 6$, $\alpha = 1.6$.}
	\label{fig:a1p6b6}
\end{figure}


\pagebreak
\begin{figure}[t!]
	\centering
	\includegraphics[width=1\linewidth]{Critical_Potential_b_6/a_1p8_b_6}
	\caption{Anti-binding state for $\beta = 6$, $\alpha = 1.8$.}
	\label{fig:a1p8b6}
\end{figure}

\begin{figure}[t!]
	\centering
	\includegraphics[width=1\linewidth]{Critical_Potential_b_6/a_2p0_b_6}
	\caption{Anti-binding state for $\beta = 6$, $\alpha = 2.0$.}
	\label{fig:a2p0b6}
\end{figure}




\end{document}